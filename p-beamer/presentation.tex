\documentclass[10pt]{beamer}

\usetheme[background=light,everytitleformat=regular,inner/sectionpage=progressbar,block=fill]{m}

\usepackage[numbers,round]{natbib}

\usepackage{booktabs}
\usepackage[scale=2]{ccicons}

\title{Fourier-Transformation}
\subtitle{Hintergrund und Anwendungen der Spektralanalyse}
\date{\today}
\author{Adrian Schrader}
\institute{Physik 4h, Herr Kuhn}

\begin{document}

\maketitle

\begin{frame}
  \frametitle{Inhaltsübersicht}
  \setbeamertemplate{section in toc}[sections numbered]
  \tableofcontents
\end{frame}

\section{Introduction}

\begin{frame}[fragile]
  \frametitle{mtheme}

  The \textsc{Metropolis} theme is a Beamer theme with minimal visual noise
  inspired by the \href{https://github.com/hsrmbeamertheme/hsrmbeamertheme}{\textsc{hsrm} Beamer
  Theme} by Benjamin Weiss.

  Enable the theme by loading

  \begin{verbatim}    \documentclass{beamer}
    \usetheme{m}\end{verbatim}

  Note, that you have to have Mozilla's \emph{Fira Sans} font and XeTeX
  installed to enjoy this wonderful typography.
\end{frame}
\begin{frame}[fragile]
  \frametitle{Sections}
  Sections group slides of the same topic

  \begin{verbatim}    \section{Elements}\end{verbatim}

  for which \textsc{Metropolis} provides a nice progress indicator \ldots
\end{frame}

\plain{Fragen?}

\begin{frame}[allowframebreaks]

  \nocite{*}
  \bibliography{../bib/bibliography}
  \bibliographystyle{abbrv}

\end{frame}

\end{document}
